% ========= common/macros.tex =========
% 常用数学命令与记号(尽量不与 amsmath 等冲突)

% 依赖的轻量宏包(只在未加载时加载)
\providecommand{\RequirePackage}[1]{\usepackage{#1}}
\RequirePackage{mathtools}  % \coloneqq, \DeclarePairedDelimiter
\RequirePackage{bbm}        % \mathbbm{1} 指示函数

% --- 数学集合与基础符号 ---
\newcommand{\RR}{\mathbb{R}}
\newcommand{\CC}{\mathbb{C}}
\newcommand{\QQ}{\mathbb{Q}}
\newcommand{\ZZ}{\mathbb{Z}}
\newcommand{\NN}{\mathbb{N}}
\newcommand{\FF}{\mathbb{F}}

% --- 线性代数常用 ---
\DeclareMathOperator{\Tr}{tr}
\DeclareMathOperator{\diag}{diag}
\DeclareMathOperator{\rank}{rank}
\DeclareMathOperator{\Span}{span}
\DeclareMathOperator{\Ker}{ker}
\DeclareMathOperator{\Img}{im}

% --- 优化与分析 ---
\DeclareMathOperator*{\argmin}{arg\,min}
\DeclareMathOperator*{\argmax}{arg\,max}
\DeclareMathOperator{\prox}{prox}
\DeclareMathOperator{\sgn}{sgn}
\DeclareMathOperator{\dom}{dom}
\DeclareMathOperator{\epi}{epi}
\DeclareMathOperator{\conv}{conv}
\DeclareMathOperator{\cl}{cl}
\DeclareMathOperator{\intr}{int}
\DeclareMathOperator{\ri}{ri}   % relative interior

% --- 概率统计 ---
\newcommand{\PP}{\mathbb{P}}
\newcommand{\EE}{\mathbb{E}}
\DeclareMathOperator{\Var}{Var}
\DeclareMathOperator{\Cov}{Cov}
\DeclareMathOperator{\Corr}{Corr}
\newcommand{\ind}{\mathbbm{1}}    % 指示函数 \ind_{A}
\newcommand{\given}{\,\middle|\,} % 条件竖线:\EE[X \given Y]

% --- 配对括号(自动伸缩)---
\DeclarePairedDelimiter{\abs}{\lvert}{\rvert}
\DeclarePairedDelimiter{\norm}{\lVert}{\rVert}
\DeclarePairedDelimiter{\paren}{\lparen}{\rparen}
\DeclarePairedDelimiter{\brak}{\lbrack}{\rbrack}
\DeclarePairedDelimiter{\set}{\lbrace}{\rbrace}
\DeclarePairedDelimiter{\ang}{\langle}{\rangle}

% 用法示例:\abs{x}, \norm{x}_2, \ang{x,y}, \set{ i : x_i>0 }

% --- 常用运算符与记号 ---
\newcommand{\defeq}{\coloneqq}     % 定义等号 :=
\newcommand{\eqdef}{\eqqcolon}     % 反向定义等号 =:
\newcommand{\ip}[2]{\left\langle #1,\,#2 \right\rangle}  % 内积
\newcommand{\dd}{\,\mathrm{d}}     % 积分微分符号
\newcommand{\Grad}{\nabla}         % 梯度
\newcommand{\Hess}{\nabla^2}       % Hessian
\newcommand{\T}{\mathsf{T}}        % 转置 ^\T
\newcommand{\HT}{\mathsf{H}}       % 共轭转置 ^\HT

% --- 取整/上取整 ---
\DeclarePairedDelimiter{\floor}{\lfloor}{\rfloor}
\DeclarePairedDelimiter{\ceil}{\lceil}{\rceil}

% --- 方便的范数下标写法 ---
\newcommand{\onenorm}[1]{\left\lVert #1 \right\rVert_{1}}
\newcommand{\twonorm}[1]{\left\lVert #1 \right\rVert_{2}}
\newcommand{\infnorm}[1]{\left\lVert #1 \right\rVert_{\infty}}
