\documentclass[12pt, a4paper, oneside]{ctexbook}
\usepackage{amsmath, amsthm, amssymb, bm, graphicx, hyperref, mathrsfs}

\usepackage{biblatex}
\usepackage{algorithm}
\usepackage{algorithmic}
\usepackage{tikz}
\usepackage{listings} % 用于代码展示




\addbibresource{ref.bib}
\title{{\Huge{\textbf{RKHS学习笔记}}}}
\author{李佶操 \\24212792}
\linespread{1.5}
\newtheorem{theorem}{定理}[section]
\newtheorem{definition}[theorem]{定义}
\newtheorem{lemma}[theorem]{引理}
\newtheorem{corollary}[theorem]{推论}
\newtheorem{example}[theorem]{例}
\newtheorem{proposition}[theorem]{命题}

\begin{document}

\maketitle

\pagenumbering{roman}
\setcounter{page}{1}

\pagenumbering{Roman}
\setcounter{page}{1}
\tableofcontents
\newpage
\setcounter{page}{1}
\pagenumbering{arabic}

\chapter{RKHS基础知识}
\section{再生核 Hilbert 空间}
本节中的Hilbert空间均指实Hilbert空间.对于复Hilbert空间,下面的对称需要变为共轭对称,其余推导相似.
\begin{definition}[核函数]
    设$X$是一个非空集合,称$X\times X$上的对称半正定函数$K:X\times X\to\mathbb{R}$为核函数.
    其中,称$K$是\textbf{对称的},若$K(x,y)=K(y,x),(\forall x,y\in X)$;称$K$是\textbf{半正定的},若对任意的有限点集$S:=\{x_1,\dots,x_k\}$,
    \begin{equation}\label{eq: Gram_matrix}
        K[S]:=\begin{pmatrix}
            K(x_1,x_1)&\cdots&K(x_1,x_k)\\
            \vdots&\ddots&\vdots\\
            K(x_k,x_1)&\cdots&K(x_k,x_k)
        \end{pmatrix}
    \end{equation}
    都是半正定的.称式\eqref{eq: Gram_matrix}中的矩阵为$K$在集合$S$上的\textbf{Gram矩阵}.
\end{definition}

\begin{theorem}\label{thm:对称半正定K(xx)>0}
    设$K$是集合$X$上的核函数,则对任意的$x,y\in X$,有
    
    (1)$K(x,x)\geq 0$;
    
    (2)$\left|K(x,y)\right|^2\leq K(x,x)K(y,y)$.
\end{theorem}

\begin{proof}
    取$S:=\{x\}$,由$K[S]$的半正定性得(1)成立.取$S':=\{x,y\}$,由$K[S]$得半正定性得(2)成立.
\end{proof}

对任意给定$x\in X$,定义$K_x$:
\begin{equation}\label{def:k_x}
    \begin{aligned}
K_x:X & \to\mathbb{R} \\
t & \mapsto K(x,t).
\end{aligned}
\end{equation}

下面我们给出再生核Hilbert空间的定义,并说明再生核Hilbert空间的再生核实际上是和某个核函数是一一对应的,因此再生核一定是对称半正定的.
\begin{definition}[再生核Hilbert空间]\label{def:RKHS}
    设$X$是一个非空集合,$\mathcal{H}$是一个Hilbert空间,其中每一个元素都是一个$X$上的实值函数. 若存在函数$K:X\times X\to\mathbb{R}$,满足以下条件:
    \begin{enumerate}
        \item $K_x\in\mathcal{H},\forall x\in X$;
        \item $f(x)=\langle f,K_x\rangle,\forall f\in\mathcal{H}$.
    \end{enumerate}
    则称$\mathcal{H}$是集合$X$上的一个再生核Hilbert空间(简记为RKHS),并称$K$为其再生核.
\end{definition}

从以上定义中,我们可以看出可再生核Hilbert空间实际上是一种特殊的Hilbert空间,其中每个元素都是一个函数,并且这个函数在任意一点的函数值都可以由再生核去重构.除此之外,结合Riesz表示定理,我们可以得到以下判断Hilbert空间是RKHS的充分必要条件.

\begin{theorem}
    设$X$是一个非空集合,$\mathcal{H}$是一个Hilbert空间,其中每一个元素都是一个$X$上的实值函数,那么$\mathcal{H}$是一个RKHS当且仅当:$\forall x\in X,f\in\mathcal{H}$都有
    \begin{equation*}
        \delta_x:f\to f(x)
    \end{equation*}
    是$\mathcal{H}$上的一个有界线性泛函.
\end{theorem}

\begin{proof}
    必要性. 设$\mathcal{H}$是一个RKHS,$K(x,y)$是其再生核,则对任意$f\in\mathcal{H},x\in X$,有:
    \begin{align*}
        |\delta_x(f)|&=|f(x)|=|\langle f,K_x\rangle| \leq \|f\|\|K_x\|,
    \end{align*}
    其中$\|\cdot\|$表示由内积诱导的范数.又因为
    \begin{equation*}
        \|K_x\|^2=\langle K_x,K_x\rangle =K(x,x)<\infty,
    \end{equation*}
    故$\delta_x\in\mathcal{H}^*$.

    充分性. 对任意$x\in X$,因为$\delta_x\in\mathcal{H}^*$,根据Riesz表示定理可知,存在$K_x\in\mathcal{H}$,使得
    \begin{equation*}
        f(x)=\delta_x(f)=\langle f,K_x\rangle,~~\forall f\in\mathcal{H}.
    \end{equation*} 
    令$K(x,y):=K_x(y)(\forall y\in\mathcal{H})$,结合$x$的任意性,容易验证$K$满足定义\ref{def:RKHS}中的条件,因此$\mathcal{H}$是一个RKHS.
\end{proof}


下面我们将证明RKHS的再生核是唯一的,并且其再生核必是一个核函数.
\begin{theorem}[RKHS决定再生核]
    再生核Hilbert空间$\mathcal{H}$的再生核是唯一的,并且是一个核函数.
\end{theorem}

\begin{proof}
    反证而设$K_1,K_2$都是$\mathcal{H}$的再生核,则对任意$x\in X$,有:
    \begin{equation*}
        \langle f,(K_1)_x-(K_2)_x\rangle=f(x)-f(x)=0,
    \end{equation*}
    由$f$的任意性可知,$(K_1)_x=(K_2)_x$.结合$x$的任意性得$K_1=K_2$.

    对任意$y\in X$,因为 $K_y\in\mathcal{H}$,令$f:=K_y$,则有
    \begin{equation*}
        K(y,x)=\langle K_y,K_x\rangle=\langle K_x,K_y\rangle=K(x,y),~~\forall x\in X,
    \end{equation*}
    其中第二个等式是因为$\mathcal{H}$是实的Hilbert空间.因此$K$是对称的.

    对任意$S=\{x_1,\dots,x_n\}$,以及$\boldsymbol{c}\in\mathbb{R}^n$,记$f:=\sum_{i=1}^n c_i K_{x_i}$,则有
    \begin{equation*}
        \boldsymbol{c}^\top K[S]\boldsymbol{c}=\sum_{i=1}^n\sum_{j=1}^nc_ic_jK(x_i,x_j)=\sum_{i=1}^n\sum_{j=1}^nc_ic_j\langle K_{x_i},K_{x_j}\rangle = \langle f,f\rangle\geq 0.
    \end{equation*}
    因此$K[S]$是半正定的. 所以$K$是核函数.
\end{proof}

前面说明了,一个RKHS的再生核存在唯一,并且是一个核函数. 那么反过来,一个核函数是否能诱导一个RKHS呢?这就是著名的 Moore-Aronszajn 定理.

\begin{theorem}[Moore-Aronszajn]\label{thm:Moore-Aronszajn}
    设 $X$ 是一个非空集合,$K:X\times X\to \mathbb{R}$是一个核函数,则存在$X$上唯一的再生核Hilbert空间以$K$作为再生核.
\end{theorem}

\begin{proof}
    定义$\mathcal{H}_0:=\operatorname{span}\{K_x:x\in X\}$, 并定义其上的内积:
    \begin{equation}
        \langle f,g\rangle:=\sum_{i=1}^m\sum_{j=1}^na_ib_jK(x_i,y_j),\quad\forall f:=\sum_{i=1}^ma_iK_{x_i}, ~g:=\sum_{j=1}^nb_jK_{y_j}\in \mathcal{H}_0,
    \end{equation}
    其中$x_i,y_j\in X~(\forall i\in[m],~j\in [n])$.  

    因为$K$是半正定的,因此存在$\Delta a_i,\Delta b_j~(i\in[m],~j\in[n])$使得
    \begin{equation}
        \sum_{i=1}^m\Delta a_i  K_{x_i}(z)=\sum_{j=1}^n\Delta b_j  K_{y_j}(z),~~\forall z\in X.
    \end{equation}
    故存在$f,g$的多种表示:
    \begin{align*}
        f&=\sum_{i=1}^ma_iK_{x_i}=\sum_{i=1}^m(a_i+\Delta a_i)K_{x_i},\\
        g&=\sum_{j=1}^nb_j K_{y_j}=\sum_{j=1}^n(b_j+\Delta b_j)K_{y_j}.
    \end{align*}

    对以上的任意两种表示有:
    \begin{align*}
        &\sum_{i=1}^m\sum_{j=1}^n(a_i+\Delta a_i)(b_j+\Delta b_j)K(x_i,y_j)-\sum_{i=1}^m\sum_{j=1}^na_ib_jK(x_i,y_j)\\
        &=\sum_{i=1}^ma_i\sum_{j=1}^n \Delta b_j K(x_i,y_j)+\sum_{j=1}^n b_j\sum_{i=1}^m\Delta a_i K(x_i,y_j)+\sum_{i=1}^m\Delta a_i\sum_{j=1}^n \Delta b_j K(x_i,y_j)\\
        &=0.
    \end{align*}
因此$\langle f,g\rangle$与$f,g$的表示无关,因此是良定的.

    显然$\langle\cdot,\cdot\rangle$满足对称性,双线性性以及$f=g$时的非负性.下面证明$\langle f,f\rangle=0$蕴含$f=0$.

    任给$f\in \mathcal{H}_0$满足$\langle f,f\rangle=0$,只需证明$f(x)=0~(\forall x\in X)$. 设$f=\sum_{i=1}^m a_i K_{x_i}$,其中$a_i\in\mathbb{R},~x_i\in X,~i=1,\dots,n.$

    考虑$S=\{x_1,\dots,x_m\}\cup \{x\}$,由$K[S]$的半正定性,对任意的$\epsilon>0$,有:
    \begin{equation*}
        \sum_{i=1}^m\sum_{j=1}^m a_i a_j K(x_i,x_j)+2\sum_{i=1}^m a_iK(x_i,x)\epsilon+\epsilon^2K(x,x)\geq 0 .
    \end{equation*}

    又因为$\sum_{i=1}^m\sum_{j=1}^m a_i a_j K(x_i,x_j)=\langle f,f\rangle =0$,取$\epsilon$充分小有
    $$f(x)=\sum_{i=1}^m a_iK(x_i,x)=0.$$
    由$x$的任意性,得$f=0$. 所以$\langle f,g \rangle$是$\mathcal{H}_0$上内积. 设$\mathcal{H}$为$\mathcal{H}_0$ 上的完备化空间(内积诱导范数意义下).但值得小心的是,$\mathcal{H}$作为$\mathcal{H}_0$的完备化空间,其中的点$f$不一定是定义在$X$上的函数,因此$\mathcal{H}$并不是我们所要找的再生核Hilbert空间.但是我们将说明,他们是等距同构的.

    对任意的$f\in\mathcal{H}$,令$\phi_f(x):=\langle f,K_x\rangle,~(\forall x\in X)$. 对任意$f,g\in\mathcal{H}$满足$\phi_f(x)=\phi_g(x),~(\forall x\in X)$, 则$\langle f-g,K_x\rangle=0$. 因此存在$h_k\in\mathcal{H}_0,h_k\to f-g$,使得$\langle f-g,  h_k\rangle=0$,因此$f-g=0$,即$f\to \phi_f$是一一对应的,即我们可以将$\mathcal{H}$中的点看作$X$上的一个函数.记$\widetilde{\mathcal{H}}:=\{\phi_f:f\in\mathcal{H}\}$,并定义其上的线性运算和内积:
    \begin{equation*}
        \phi_{f}+\phi_{g}:=\phi_{f+g},\quad\lambda\phi_{f}:=\phi_{\lambda f},\quad\langle\phi_{f},\phi_{g}\rangle_{\widetilde{\mathcal{H}}}:=\langle f,g\rangle,\quad\forall f,g\in \mathcal{H},\quad\lambda\in\mathbb{R}.
    \end{equation*}
    显然$T:f\to\phi_f$是映$\mathcal{H}$到$\widetilde{\mathcal{H}}$的一个等距同构,满足
    \begin{equation*}
        \phi_{f}=f,\forall f\in \mathcal{H}_{0},\quad\quad\langle\phi_{f},K_x\rangle_{\widetilde{\mathcal{H}}}=\langle f,K_x\rangle=\phi_{f}(x),~~\forall f\in \mathcal{H}.
    \end{equation*}
因此$\widetilde{\mathcal{H}}$是以$K$为核的Hilbert空间.

最后证其唯一性.反证而设$\mathcal{H},\mathcal{G}$都是以$K$为核的Hilbert空间,由再生核等式有
\begin{equation*}
    \langle K_x,K_y\rangle_{\mathcal{H}}=K(x,y)=\langle K_x,K_y\rangle_{\mathcal{G}}
\end{equation*}
结合内积的双线性性可知,对任意$f,g\in \mathcal{H}_0$有
\begin{equation*}
    \langle f,g\rangle_{\mathcal{H}}=\langle f,g\rangle_{\mathcal{G}}
\end{equation*}
因此$\mathcal{H}_0\subset \mathcal{G}$,结合$\mathcal{G}$的完备性有$\mathcal{H}$是$\mathcal{G}$的闭子空间. 由正交分解定理可知$\mathcal{G}=\mathcal{H}\oplus\mathcal{H}^{\perp}$. 任取$f\in\mathcal{H}^\perp$,有
\begin{equation*}
    f(x)=\langle f,K_x\rangle_{\mathcal{G}}=0,~~\forall x\in X.
\end{equation*}
因此$\mathcal{H}^{\perp}=\{0\}$,即$\mathcal{G}=\mathcal{H}$. 唯一性获证.
\end{proof}

\textbf{注:}通常记以$K$为核的Hilbert空间为$\mathcal{H_K}$.

从上面的叙述中,我们可以看到再生核Hilbert空间和函数$K_x$息息相关,现在,我们将$K_x$的定义推广到更一般的情形,即$K_x$不再是一个函数,而是落在一个Hilbert空间中的一个元素. 称$\Phi:X\to\mathcal{H}$为$X$上的特征映射,其中$\mathcal{H}$是一个实的Hilbert空间. 令$K(x,y):=\langle \Phi(x),\Phi(y)\rangle$,显然$K(x,y)$是一个核函数,由 Moore-Aronszajn 定理\ref{thm:Moore-Aronszajn}可知 $K$诱导了一个以其为核的Hilbert空间$\mathcal{H_K}$. 一般而言,$\mathcal{H}$和$\mathcal{H_K}$是不同的,实际上,$\mathcal{H_K}$是$X$上某些函数的集合, 但$\mathcal{H}$未必是. 而下面的定理说明$\mathcal{H_K}$和$\mathcal{H}$的一个子空间是等距同构的.

\begin{theorem}
    给定非空集合$X$,实Hilbert空间$\mathcal{H}$以及映射$\Phi:X\to \mathcal{H}.$设$\mathcal{H}_\mathrm{\Phi}$是$\mathcal{H}_{0}:=\operatorname{span}\{\Phi(x)|x\in X\}$在$\mathcal{H}$中的完备化空间.又设$K$是$\Phi$的核函数,那么,存在等距同构$T:\mathcal{H}_\Phi\to\mathcal{H_K}$,使得$T\Phi ( x) = K_x$, $\forall x\in X.$
\end{theorem}

\begin{proof}
    考虑映射$T:\mathcal{H}_0\to\mathcal{H_K}$:
    \begin{equation*}
        \sum_{i=1}^m c_i\Phi(x_i)\mapsto\sum_{i=1}^mc_iK_{x_i}
    \end{equation*}
    其中$x_i\in X$, $c_i\in \mathbb{R}$, $i= 1, \dots , m.$根据核函数和 RKHS 的定义,我们有
$$\left\|\sum_{i=1}^{m}c_{i}K_{x_{i}}\right\|_{\mathcal{H_K}}^{2}=\sum_{i=1}^{m}\sum_{j=1}^{m}c_{i}c_{j}K(x_{i},x_{j})=\sum_{i=1}^{m}\sum_{j=1}^{m}c_{i}c_{j}\langle\Phi(x_{i}),\Phi(x_{j})\rangle_{\mathcal{H}}=\left\|\sum_{i=1}^{m}c_{i}\Phi(x_{i})\right\|_{\mathcal{H}}^{2}.$$

所以$\sum_{i=1}^{m}c_{i}K_{x_{i}}=0$ 当且仅当$\sum_{i=1}^{m}c_{i}\Phi(x_{i})=0.$ 这说明$T$是单射,并且$T$是等距线性映射.分别将$\mathcal{H}_0$和 $\operatorname{span}\{K_x|x\in X\}$完备化,得到 Hilbert 空间$\mathcal{H}_\Phi$和$\mathcal{H_K}.$显然$T$可以唯一地推广到$\mathcal{H}_\Phi$上,使得$T:H_\Phi\to\mathcal{H_K}$ 是一个等距同构.易见,$T\Phi ( x) = K_x$, $\forall x\in X.$

\end{proof}
\textbf{注:} 这里的$\operatorname{span}\{K_x|x\in X\}$完备化后并不是$\mathcal{H_K}$,实际上是一个与$\mathcal{H_k}$同构的Hilbert空间.但是根据同构的复合运算,上述证明依然合理.


\section{Mercer 核情形的再生核Hilbert空间}
以$K:X\times X \to \mathbb{R}$为再生核的Hilbert空间$\mathcal{H_K}$是$X$上一些函数的集合,我们自然的希望将$\mathcal{H_K}$嵌入到某个熟知的Hilbert空间中,并且这个空间中的元素也是$X$上的函数.一个自然的选择是实的$L^2(X,\mu)$,这就需要对$X$赋予一个测度结构,才能定义其上的可测函数以及积分. 我们设$X$是一个距离空间,$\mu$是定义在$X$的Borel集上的一个Borel 测度.

\textbf{注:} $X$是距离空间就自然地定义了其上的拓扑结构,因此这里的Borel集是指这个由距离诱导的拓扑,其全体开集构成的$\sigma$代数的元素.即任意开集、闭集的任意交并. 因此简单函数、连续函数显然都是Borel可测函数.


在再生核Hilbert空间理论中我们通常假设$\mu$是$X$上有限的,即$\mu(X)<\infty$. 此外,我们用$\operatorname{supp} \mu$表示$\mu$的支集,其定义为:
\begin{equation*}
    \mathrm{supp}\mu:=\{x\in X|\mu(B(x,\epsilon))>0,\forall\epsilon>0\},
\end{equation*}
对于Lebesgue测度$\lambda$,显然$\operatorname{supp}\lambda=X$.
\begin{example}
    Dirac 测度(Dirac measure)是在可测空间 $(X, \mathcal{F})$ 上以某个点 $x_0 \in X$ 为“全部质量”的概率测度,记作 $\delta_{x_0}$,定义为
$$\delta_{x_0}(A) = \begin{cases}
1, & x_0 \in A, \\
0, & x_0 \notin A,
\end{cases} \quad A \in \mathcal{F}.$$

则$\operatorname{supp}\delta_{x_0}=\{x_0\}.$
\end{example}

为了使$K_x\in L^2(X,\mu)$,我们通常假设$X$是紧的,并且$K$是连续的,这就是Mercer核的定义:
\begin{definition}
    设$X$是一个紧的距离空间,$K:X\times X\to\mathbb{R}$是一个对称半正定函数,若$K$是连续的,则称之为$X$上一个Mercer核.
\end{definition}


下面,我们将介绍由Mercer核诱导的再生核 Hilbert 空间$\mathcal{H_K}$的嵌入定理.
\begin{theorem}
    设$K$是$X$上一个 Mercer 核,$\mu$是$X$上有限的Borel测度,则
    
(1)恒同映射$I_K:\mathcal{H_K}\to C(X)$是连续的,且$\|I_K\|\leq C_K$,其中$C_K:=\sup_{x\in X}\sqrt{K(x,x)}.$

(2)恒同映射$I_\mu:C(X)\to L^2(X,\mu)$是连续的,且$\|I_\mu\|\leq C_\mu:=\sqrt{\mu(X)}.$
\end{theorem}

\begin{proof}
    记 $C_K := \sup_{x \in X} \sqrt{K(x, x)}$. 那么, $\forall f \in \mathcal{H}_0 := \text{span}\{K_x | x \in X\}$, 易见 $f \in C(X)$. 由于 $f(x) = \langle f, K_x  \rangle$, 且 $\|K_x\|_{\mathcal{H_K}}^2 = \langle K_x, K_x \rangle = K(x, x) \leq C_K^2$, 我们有
$$|f(x)| \leq \|f\|_{\mathcal{H_K}} \|K_x\|_{\mathcal{H_K}} \leq C_K \|f\|_{\mathcal{H_K}}.$$
即$\|f\|_{C(X)}\leq C_K\|f\|_{\mathcal{H_K}}$.显然$\mathcal{H}_0$的完备化空间仍是$C(X)$的子空间,且$\|f\|_{C(X)}\leq$
$C_{K}\|f\|_{\mathcal{H_K}}$对$f\in\mathcal{H_K}$仍成立.故(1)成立.

对任意的$f\in C(X)$,显然有

$$\int_X|f(x)|^2d\mu(x)\leq\|f\|_{C(X)}^2\mu(X).$$
故$f\in L^2(X,\mu)$,且恒同映射$I_\mu:C(X)\to L^2(X,\mu)$满足$\|I_\mu\|\leq C_\mu:=\sqrt{\mu(X)}$.所以(2)成立.
\end{proof}

在研究完$\mathcal{H}_k$的嵌入关系后. 进一步,我们希望了解$\mathcal{H_K}$是$L^2(X,\mu)$的一个怎样的子空间,它等于$L^2(X,\mu)$吗? 在$L^2(X,\mu)$中稠密吗?是有限维子空间吗?为了研究这些问题,我们将介绍著名的Mercer定理.

\subsection{由核函数诱导的积分算子}
下面均考虑$X$是一个紧的距离空间, $\mu$是$X$上一个有限的Borel测度.$K:X\times X\to\mathbb{R}$是对称的连续函数.令$T_K$表示由核$K(x,y)$诱导的积分算子,即
\begin{equation*}
    T_Kf(x):=\int_X f(y)K(x,y)\mathrm{d}\mu(y),\quad f\in L^2(X,\mu)
\end{equation*}
下断言$T_Kf\in L^2(X,\mu)$.

因为$X$紧,故$X\times X$紧,结合$K$的连续性,有$\operatorname{Im}(K)$紧,因此有界. 固有
\begin{equation*}
    \left|T_Kf(x)\right|\leq \|f\|_2\left(\int_X |K(x,y)|^2\mathrm{d}\mu(y)\right)^{1/2}\leq \|f\|_2\|K\|_\infty \mu(X).
\end{equation*}
结合$\mu(X)<\infty$,有$T_Kf\in L^2(X,\mu)$.

除此之外,算子$T_K$还是紧的自伴算子,即$\langle T_Kf,f\rangle=\langle f,T_Kf\rangle$. 紧性由Arzelà–Ascoli 定理即可得到,而自伴性由$K$的对称性即可得到.而下面的定理将说明对于Mercer核而言,积分算子的半正定性,即$\langle T_Kf,f\rangle\geq 0~(\forall f\in L^2(X,\mu))$,可由积分核$K$的半正定性得到.
\begin{theorem}[基于Mercer核的积分算子的半正定性]\label{thm:基于Mercer核的积分算子的半正定性}
    设 $X$ 是一个紧的距离空间, $\mu$ 是 $X$ 上一个有限的 Borel 测度, $K: X \times X \to \mathbb{R}$ 是一个连续的对称函数, 那么

(1) 若 $K$ 是半正定函数, 则$T_K$是半正定的,即
\begin{equation}\label{eq:正定的必要条件}
    \int_X \int_X f(x)f(y)K(x,y)d\mu(x)d\mu(y) \geq 0, \quad \forall f \in L^2(X,\mu).
\end{equation}

(2) 反之, 若式\eqref{eq:正定的必要条件}成立, 且 $\text{supp}\mu = X$, 则 $K$ 是半正定函数.
\end{theorem}
\begin{proof}
    (1) 因为$K$在紧集$X\times X$上连续,因此一致连续,即对任意$\epsilon>0$,存在$\delta>0$,当$|x_1-x_2|^2+|y_1-y_2|^2\leq 2\delta^2$时,有
    \begin{equation*}
        |K(x_1,y_1)-K(x_2,y_2)|<\epsilon.
    \end{equation*}
    
    又因为$X$紧,所以$M$是完全有界的,于是对上述$\delta$存在有穷$\delta-$网$\{x_1,\dots,x_n\}$, 即$X\subset\bigcup _{i=1}^n B(x_i,\delta)$.记
    \begin{equation*}
        E_1:=B(x_1,\delta),\quad E_i:=B(x_i,\delta)\setminus\bigcup_{j=1}^{i-1}B(x_j,\delta),\quad i=2,\cdots,n.
    \end{equation*}
    易见$\{E_i\}_{i=1}^n$是互不相交的 Borel 集,且$X=\bigcup_{i=1}^nE_i.$于是对任意$f\in L^2(X,\mu)$,有
    \begin{equation*}
\int_X\int_Xf(x)f(y)K(x,y)\mathrm{d}\mu(x)\mathrm{d}\mu(y)=\sum_{i=1}^n\sum_{j=1}^n\int_{E_i}\int_{E_j}f(x)f(y)K(x,y)\mathrm{d}\mu(x)\mathrm{d}\mu(y).
    \end{equation*}
    将$K(x,y)=K(x_{i},y_{j})-[K(x,y)-K(x_{i},y_{j})]$代入上式,得

$$\int_X\int_Xf(x)f(y)K(x,y)\mathrm{d}\mu(x)\mathrm{d}\mu(y)\geq\sum_{i=1}^n\sum_{j=1}^nK(x_i,x_j)c_ic_j-\epsilon\Big(\int_X|f(x)|\mathrm{d}\mu(x)\Big)^2.$$

其中

$$c_i:=\int_{E_i}f(x)\mathrm{d}\mu(x),\quad i=1,\cdots,n.$$

由$K(x,y)$的正定性可知上式第一项非负.而对于第二项,由于$\mu(X)<\infty$,所以
$f\in L^{2}(X,\mu)\subset L^{1}(X,\mu)$,因而其当$\epsilon\to0$时,趋于0.由$\epsilon$的任意性,可知\eqref{eq:正定的必要条件}成立.

(2)设\eqref{eq:正定的必要条件}成立. 任取 $X$ 是 $n$ 个不同的点 $x_1,\dots,x_n$, 取$ \delta'\in (0,\delta)$, 使得球 $\{B(x_i, \delta')\}^{n}_{i=1}$ 互不相交.由$\operatorname{supp}\mu=X$ 可知$\mu(B(x_i,\delta'))>0~(\forall i\in[n])$.对任意$c_1,\dots, c_n \in \mathbb{R},$ 作函数
$$f(x) := \sum^{n}_{i=1} \frac{c_i}{\mu(B(x_i, \delta'))} \mathcal{X}_{B(x_i, \delta')}.$$
其中$\mathcal{X}$表示集合的示性函数. 由于 $f$ 是 $X$ 上简单函数, 故 Borel 可测. 代入\eqref{eq:正定的必要条件}得
\begin{equation}\label{eq:微分中值定理}
    \sum^{n}_{i=1} \sum^{n}_{j=1} \frac{c_i}{\mu(B(x_i, \delta'))} \frac{c_j}{\mu(B(x_j, \delta'))} \int_{B(x_i, \delta')} \int_{B(x_j, \delta')} K(x, y) \mathrm{d}μ(x) \mathrm{d}μ(y) \geq  0.
\end{equation}
同理,将$K(x,y)=K(x_{i},y_{j})-[K(x,y)-K(x_{i},y_{j})]$代入上式,结合$K$的连续性,得
\begin{equation}\label{eq: K的半正定性转}
    \sum^{n}_{i=1} \sum^{n}_{j=1} c_ic_jK(x_i,x_j) \geq  \epsilon.
\end{equation}
由$\epsilon$的任意性得$K$是半正定的.

\end{proof}

\textbf{注:}式\eqref{eq: K的半正定性转}不能由式\eqref{eq:微分中值定理}结合Lebesgue微分中值定理直接得到,因为对于Lebesgue 微分中值定理依赖于Vitali 覆盖性质,但是对于一般的Borel测度,vitali性质不一定满足. 而对于该定理,完全可以绕过微分中值定理,利用$K$的连续性即可.

\begin{lemma}[SejdinovicGretton2012RKHSNotes]\label{lemma:Sejdinovic}
设 $\mathcal{F}$ 为一Hilbert空间,$T:\mathcal{F}\to\mathcal{F}$ 为半正定的紧、自伴算子.
则存在一个至多可数的规范正交基$\{\phi_j\}_{j\in J}\subset\mathcal{F}$ 与一列实数
$\{\lambda_j\}_{j\in J}$,满足
\[
\lambda_1\ge \lambda_2\ge \cdots \geq  0,\qquad \lambda_j\to 0,
\]
使得对任意 $f\in\mathcal{F}$,
\[
Tf=\sum_{j\in J}\lambda_j\,\langle f,\phi_j\rangle_{\mathcal{F}}\,\phi_j .
\]
\end{lemma}


\subsection{Mercer定理}
根据前面的分析,我们已经知道对于mercer核诱导的算子$T_K$是自伴半正定的紧算子.根据引理\ref{lemma:Sejdinovic},我们取$T_K$的一组特征向量作为$L^2(X,\mu)$的规范正交基,即存在$\{\phi_j\}_{j\in J}$是$L^2(X,\mu)$的一组规范正交基,满足

$$T_K\phi_j=\lambda_j\phi_j,\quad j\in J,$$
其中 $J$ 是有限集或 $J=\mathbb{N}_+,\lambda_1\geq\lambda_2\geq\cdots,$且当$J=\mathbb{N}_+$时,有 $\lambda_j\to0~(j\to\infty).$

下面,我们将证明著名的Mercer定理,为此,我们先给出以下几个引理.
\begin{lemma}[$L^2$意义下相等的等价性]\label{lemma: L2中等价性}
    设 $(X,\mathcal{A},\mu)$ 为测度空间,$f,g\in L^2(X,\mu)$.则以下两条等价:
\begin{enumerate}
    \item $\|f-g\|_{L^2}=0$(即 $f=g$ 在 $L^2$ 意义下,亦即几乎处处相等);
    \item 对任意 $h\in L^2(X,\mu)$,都有
    \[
        \langle f,h\rangle=\langle g,h\rangle,
    \]
    其中 $\langle u,v\rangle:=\int_X u\,\overline{v}\,d\mu$ 为 $L^2$ 的内积.
\end{enumerate}

\end{lemma}
\begin{proof}
(1)$\Rightarrow$(2):
由柯西–施瓦茨不等式,
\[
\big|\langle f,h\rangle-\langle g,h\rangle\big|
=\big|\langle f-g,h\rangle\big|
\le \|f-g\|_{L^2}\,\|h\|_{L^2}=0,
\]
故对一切 $h\in L^2$ 有 $\langle f,h\rangle=\langle g,h\rangle$.

(2)$\Rightarrow$(1):
令 $h=f-g\in L^2$,由(2) 得
\[
0=\langle f,h\rangle-\langle g,h\rangle
=\langle f-g,f-g\rangle
=\|f-g\|_{L^2}^2,
\]
从而 $\|f-g\|_{L^2}=0$.证毕.
\end{proof}


\begin{lemma}[{\(\{\phi_j\}_{j\in J}\) 的连续性}]\label{lemma: phi_j连续性}
设 $K$ 是非空集合 $X$ 上的一个 Mercer 核,$\{\phi_j\}_{j\in J}$ 如上所述,那么,
对于任意 $j\in J$,当 $\lambda_j>0$ 时,有 $\phi_j$在$X$上一致连续.
\end{lemma}

\begin{proof}
    当  $\lambda _{j}> 0$ 时,有 $\phi_j=\lambda_j^{-1}T_{K}\phi_{j}.$ 所以,我们只需证明 $T_K\phi_{j}$ 的连续性.对任意的 $x_1,x_2\in X$,有
$$\begin{aligned}|T_{K}\phi_{j}(x_{1})-T_{K}\phi_{j}(x_{2})|&\leq\int_{X}|\phi_{j}(y)||K(x_{1},y)-K(x_{2},y)|d\mu(y)|\\&\leq\|\phi_j\|_{L^2(X,\mu)}\Big(\int_X|K(x_1,y)-K(x_2,y)|^2d\mu(y)|\Big)^{1/2}.\end{aligned}$$
由于$K$ 在 $X\times X$ 上一致连续,结合上式可知:$T_K\phi_j$ 在 $X$ 上一致连续.
\end{proof}

\begin{lemma}[Dini]\label{lemma:Dini}
设 $X$ 是一个紧的距离空间,$\{u_k\}_{k\in\mathbb{N}}$ 是 $X$ 上一列连续的非负函数,
使得级数 $\sum_{k=1}^\infty u_k(x)$ 在 $X$ 上处处收敛于函数 $f(x)$,且 $f$ 是 $X$ 上连续函数.
那么 $\sum_{k=1}^\infty u_k(x)$ 在 $X$ 上一致收敛于 $f(x)$.
\end{lemma}

\begin{proof}
记 $S_n(x):=\sum_{k=1}^n u_k(x)$ 为其部分和.若 $S_n$ 不一致收敛于 $f$,
则必存在 $\epsilon_0>0$,以及自然数列的子列 $\{k_n\}$ 和点列 $\{x_n\}\subset X$,
使得 $|S_{k_n}(x_n)-f(x_n)|\ge \epsilon_0$.
根据 $\{u_k\}_{k\in\mathbb{N}}$ 的非负性,便有
\[
f(x_n)-S_{k_n}(x_n)\ge \epsilon_0.
\]
从而对任意的 $m\in\mathbb{N}$,当 $k_n\ge m$ 时,有
\[
f(x_n)-S_m(x_n)\ge f(x_n)-S_{k_n}(x_n)\ge \epsilon_0.
\]

由于 $X$ 是紧的,故 $\{x_n\}$ 有收敛的子列.不妨设 $x_n\to x_0\in X$ ($n\to\infty$).
于上述不等式中令 $n\to\infty$,得
\[
f(x_0)-S_m(x_0)\ge \epsilon_0.
\]
这与 $S_n(x_0)\to f(x_0)$ 矛盾.故假设不成立.
\end{proof}



\begin{theorem}[Mercer 定理]\label{thm:mercer}
设 $K$ 是非空集合 $X$ 上的一个 Mercer 核,$\mu$ 是定义在 $X$ 上一个有限的 Borel 测度,且满足 $\mathrm{supp}\,\mu = X$.
设 $\{\phi_j\}_{j\in J}$, $\{\lambda_j\}_{j\in J}$ 如上所述,则任意$x,y\in X$,级数$\sum_{j\in J} \lambda_j \phi_j(x)\phi_j(y)$逐点绝对收敛,并且在一致收敛的意义下,有
\begin{equation}\label{eq:mercer}
    K(x,y) = \sum_{j\in J} \lambda_j \phi_j(x)\phi_j(y), 
    \qquad \forall x,y \in X.
\end{equation}
\end{theorem}


\begin{proof}
     对任意$n\in\mathbb N_+,$ 满足 $\{1,\dots,n\} \subset J,$ 记
$$K_{n}(x,y) := \sum_{j=1}^{n} \lambda_{j} \phi_{j}(x) \phi_{j}(y).$$
首先我们证明: 对任意$ x,y \in X$, 有
$$
K_{n}(x,x) \leq K(x,x)
$$

事实上,对任意$f\in L^2(X,\mu),$在$L^2$意义下有
\begin{equation*}
    f=\sum_{j\in J}\langle f,\phi_j\rangle\phi_j,\quad T_Kf=\sum_{j\in J}\langle f,\phi_j\rangle\lambda_j\phi_j.
\end{equation*}
根据引理\ref{lemma: L2中等价性}可知
\begin{align*}
    \int_{X}\int_{X}K(x,y)f(x)f(y)d\mu(x)d\mu(y) & =\langle T_{K}f,f\rangle=\sum_{j\in J}\lambda_{j}|\langle f,\phi_{j}\rangle|^{2}, \\
\int_{X}\int_{X}K_{n}(x,y)f(x)f(y)d\mu(x)d\mu(y) & =\sum_{j=1}^n\lambda_j|\langle f,\phi_j\rangle|^2.
\end{align*}
故
\begin{equation*}
    \int_X\int_X[K(x,y)-K_n(x,y)]f(x)f(y)d\mu(x)d\mu(y)\geq0.
\end{equation*}
根据引理\eqref{lemma: phi_j连续性}可知$K-K_n$是一个连续的对称函数.故由定理\ref{thm:基于Mercer核的积分算子的半正定性}可知$K-K_n$是核函数.由定理\ref{thm:对称半正定K(xx)>0}有,对任意$x\in X$,有$K(x,x)-K_n(x,x)\geq 0$.


然后我们证明函数项级数$\sum_{j\in J} \lambda_j \phi_j(x)\phi_j(y)$逐点绝对收敛. 

因为$K$是紧集$X$上的连续函数因此有界,记$M = \sup_{x\in X} K(x,x)$. 则对任意的$n\in \mathbb{N}_+\cap J$,以及$x,y\in X$,有
\begin{align}\label{eq:K_n的有界性}
    \sum_{j=1}^n |\lambda_j\phi_j(x)\phi_j(y)|&\leq \left(\sum_{j=1}^n\lambda_j|\phi_j(x)|^2\right)^{1/2}\left(\sum_{j=1}^n\lambda_j|\phi_j(y)|^2\right)^{1/2} \nonumber\\
    &\leq \sqrt{K(x,x)K(y,y)}\leq M .
\end{align}
因此$\sum_{j\in J} \lambda_j \phi_j(x)\phi_j(y)$逐点绝对收敛.
逐点绝对收敛一定逐点收敛,记
$$\widetilde{K}(x,y):=\sum_{j\in J}\lambda_j\phi_j(x)\phi_j(y)=\lim_{n\to\infty}K_n(x,y).$$

接下来,我们证明对任意$x\in X$,以及几乎处处的$y\in X$,有$K(x,y)= \widetilde{K}(x,y).$

事实上,对任意$\phi_j,n\in\mathbb{N}_+$有
\begin{align*}
    \langle K_x,\phi_j\rangle_{L^2 } = \lambda_j \phi_j(x) = \int_{X}K_n(x,y)\phi_j(y)\mathrm{d}\mu (y).
\end{align*}
又根据式\eqref{eq:K_n的有界性}有$|K_n(x,y)|\leq \sqrt{K(x,x)K(y,y)}.$,因此,对任意给定的$x\in X$,有
\begin{equation*}
    \int_{X} \sqrt{K(x,x)K(y,y)}\cdot |f(y)|\mathrm{d}\mu(y)\leq M \|f\|_1\leq M\|f\|_2 \mu(X)<\infty.
\end{equation*}
故由Lebesgue控制收敛定理可知,
\begin{equation*}
    \langle K_x,\phi_j\rangle_{L^2} = \lim_{n\to\infty}\int_{X}K_n(x,y)\phi_j(y)\mathrm{d}\mu (y) = \int_{X}\widetilde{K}(x,y)\phi_j(y)\mathrm{d}\mu(y) = \langle \widetilde{K}_x,\phi_j\rangle_{L^2}
\end{equation*}
由内积的连续性,对任意$f\in L^2$有
\begin{equation*}
    \langle K_x,f\rangle_{L^2}=\langle \widetilde{K}_x,f\rangle_{L^2}
\end{equation*}
因此,对任意$x\in X$,以及几乎处处的$y\in X$,有$K(x,y)= \widetilde{K}(x,y).$


最后,我们证明$K_n(x,y)$一致收敛到$\widetilde{K}(x,y)$,有了一致收敛后,我们就可以有$\widetilde{K}(x,y)$的连续性.再结合$K,\widetilde{K}$的连续性以及$K = \widetilde{K}~\operatorname{a.e.} ~(x,y)\in X\times X$,即可得$K=\widetilde{K},~~\forall (x,y)\in X\times X,$并且$K_n$一致收敛到$K.$

为此,我们先证,对任意给定的$x\in X$,$K_n(x,y)$对$y\in Y$一致收敛.
因为
\begin{align*}
    \sup_{y\in Y} |\sum_{k=m}^n \lambda_k\phi_k(x)\phi_k(y)|&\leq \sup_{y\in Y}\left( \sum_{k=m}^n\lambda_k|\phi_k(x)|^2\right)^{1/2}\left( \sum_{k=m}^n\lambda_k|\phi_k(y)|^2\right)^{1/2}\\
    &\leq \sqrt M \left( \sum_{k=m}^n\lambda_k|\phi_k(x)|^2\right)^{1/2}\to 0.
\end{align*}
因此,对任意给定的$x\in X$,$K_n(x,y)$对$y\in Y$一致收敛于$\widetilde{K}(x,y).$故对任意给定的$x$,$\widetilde{K}(x,y)$连续,所以对任意的$(x,y)\in X\times X$,有$K(x,y)=\widetilde{K}(x,y).$

此外,$\{K_n(x,x)\}$是单调不减的连续非负函数列,并且逐点收敛到$\widetilde{K}(x,x)=K(x,x)$.由Dini定理\ref{lemma:Dini}可知$K_n(x,x)$在$X$上一致收敛到$K(x,x).$即
\begin{equation*}
    \sup_{x\in X}\sum_{k=m}^n \lambda_k|\phi_k(x)|^2\to 0.
\end{equation*}
因此,有
\begin{align*}
    \sup_{x\in X}\sup_{y\in Y} |\sum_{k=m}^n \lambda_k\phi_k(x)\phi_k(y)|&\leq \sup_{x\in X}\sup_{y\in Y}\left( \sum_{k=m}^n\lambda_k|\phi_k(x)|^2\right)^{1/2}\left( \sum_{k=m}^n\lambda_k|\phi_k(y)|^2\right)^{1/2}\\
    &\leq \sqrt M \sup_{x\in X}\left( \sum_{k=m}^n\lambda_k|\phi_k(x)|^2\right)^{1/2}\to 0.
\end{align*}
\end{proof}
\noindent\textbf{注:}Mercer定理给出了核函数的另一种特征映射:
\begin{equation*}
    K(x,y) =\langle \sum_{j\in J}\sqrt{\lambda_j}\phi_j(x),\sum_{j\in J}\sqrt{\lambda_j}\phi_j(y)\rangle.
\end{equation*}



\begin{corollary}
级数 $\sum_{k\geq 1}\lambda_k$ 收敛,并且有
\[
\sum_{k\geq 1}\lambda_k
= \int_X K(x,x)\,d\mu(x)
\;\leq\; \mu(X)\,C_K^2.
\]
此外,对所有 $k\geq 1$,成立
\[
\lambda_k \;\leq\; \frac{\mu(X)\,C_K^2}{k}.
\]
\end{corollary}

\begin{proof}
在Mercer定理\ref{thm:mercer}中取 $x=y$,得到
\[
K(x,x)=\sum_{k\geq 1}\lambda_k \phi_k(x)^2.
\]
两边在 $X$ 上积分,并由Lebesgue控制收敛定理得
\[
\sum_{k\geq 1}\lambda_k\int_X \phi_k(x)^2\,d\mu(x)
=\int_X K(x,x)\,d\mu(x)
\;\leq\;\mu(X)\,C_K^2.
\]
由于 $\{\phi_1,\phi_2,\dots\}$ 是一组正交归一基,所以
\(\int_X \phi_k(x)^2\,d\mu(x)=1\),
于是第一个结论成立.

对于第二个结论,由$\{\lambda_k\}$单调不增,有
\[
k\,\lambda_k \;\leq\;\sum_{j=1}^k \lambda_j
\;\leq\;\mu(X)\,C_K^2,
\]
即
\(\lambda_k \leq \mu(X)\,C_K^2/k\).
\end{proof}



\subsection{$\mathcal{H_K}$与$L^2(X,\mu)$的关系}
由于$0\leq \lambda_j\leq \lambda_1~(\forall j\in J)$
,记$J_+:=\{j\in J:\lambda_j>0\}$,定义
\begin{equation*}
    \mathcal{H}_{K,\mu}:=\{f=\sum_{j\in J^+}\sqrt{\lambda_j}c_j\phi_j:\{c_j\}\in\ell^2(J_+)\},
\end{equation*}
其中求和是在$L^2$意义下的,由于
\begin{equation*}
    \|f\|_2 = \sum_{j\in J_+}\lambda_j c_j^2\leq \lambda_1\sum_{j\in J_+}c_j^2<\infty,
\end{equation*}
因此求和在$L^2(X,\mu)$范数意义下是收敛的,因此上述定义是合理的. 定义其上的内积:
\begin{equation}\label{eq: H_Kmu 内积}
    \langle f , g \rangle_{\mathcal{H}_{K,\mu}}
:= \sum_{j\in J_+} a_j\,b_j,
\qquad
\forall\, 
f := \sum_{j\in J} \sqrt{\lambda_j}\, a_j\, \phi_j,\quad
g := \sum_{j\in J} \sqrt{\lambda_j}\, b_j\, \phi_j \in \mathcal{H}_{K,\mu}.
\end{equation}
容易验证它是一个内积,下证明完备性.任取$\mathcal{H}_{K,\mu}$的Cauchy列$\{f_n:=\sum_{j\in J_+}\sqrt{\lambda_j}c^{(n)}_j\phi_j\}$,则有
\begin{equation*}
    \|f_m - f_n\|_{\mathcal{H}_{K,\mu}} = \|c^{(m)}-c^{(n)}\|_{\ell^2(J_+)}\to 0.
\end{equation*}
由$\ell^2(J_+)$的完备性可知$c^{(n)}\to c\in\ell^2(J_+).$令$f_0:=\sum_{j\in J^+}c_j\phi_j\in\mathcal{H}_{K,\mu}$,并且有$f_n\to f_0~(\mathcal{H}_{K,\mu})$,因此$\left(\mathcal{H}_{K,\mu},\langle\cdot,\cdot\rangle_{\mathcal{H}_{K,\mu}}\right)$完备.

定义$L^2(X,\mu)$的子空间
\begin{equation*}
    L^2(X,\mu)_+:=\{f\in L^2(X,\mu):\langle f,\phi_j\rangle = 0,\forall j\in J\backslash J_+\},
\end{equation*}
事实上,还有
\begin{equation*}
    L^2(X,\mu) = L^2(X,\mu)_+ \oplus \operatorname{Ker}(T_K).
\end{equation*}

满足\(T_k = T_k^{1/2}\circ T_k^{1/2}\)的算子 \(T_k^{1/2}\)定义为
\begin{equation}\label{eq:算子T_k1/2}
    T_K^{1/2}f:=\sum_{j\in J_+}\sqrt{\lambda_j}a_j\phi_j,\qquad \forall f=\sum_{j\in J}a_j\phi_j\in L^2(X,\mu).
\end{equation}
事实上,容易得到,对任意$f\in L^2(X,\mu)_+$,有$T_K^{1/2}f\in \mathcal{H}_{K,\mu}$,并且满足以下等式:
\begin{equation}\label{eq:等距同构}
    \|T_K^{1/2}f\|_{\mathcal{H}_{K,\mu}}^2 = \sum_{j\in J_+}|a_j|^2 = \|f\|^2_{L^2(X,\mu)}.
\end{equation}

下面证明$\mathcal{H}_{K,\mu}$就是我们所要找的$\mathcal{H_K}$,并且$T_K^{1/2}$是$ L^2(X,\mu)_+$与$\mathcal{H}_{K,\mu}$之间的一个等距同构.

\begin{theorem}\label{thm:Mercer-isometry}
设 $K$ 是一个 Mercer 核,则
\[
\mathcal{H}_{K,\mu}\subset C(X), \qquad \mathcal{H}_{K,\mu}=\mathcal {H_K},
\]
并且由式\eqref{eq:算子T_k1/2} 所定义的算子
\[
T_{K}^{1/2}:~L^{2}(X,\mu)_{+}\longrightarrow \mathcal{H}_{K,\mu}
\]
是一个等距同构.
\end{theorem}

\begin{proof}
    对任意$f = \sum_{j\in J^+}\sqrt{\lambda_j}a_j\phi_j\in \mathcal{H}_{K,\mu}$,$\{a_j\}\in \ell^2(J_+)$.因此对任意$x\in X$有
    \begin{align*}
        \sup_{x\in X} \left|\sum_{k=m}^n \sqrt{\lambda_k}a_k\phi_k(x)\right|&\leq \sup_{x\in X}\left(\sum_{k=m}^na_k^2\right)^{1/2}\left(\sum_{k=m}^n\lambda_k\phi^2_k(x)\right)^{1/2}\\
        &\leq C_K\left(\sum_{k=m}^na_k^2\right)^{1/2}\to 0,
    \end{align*}
其中$C_K = \sup_{x\in X}K(x,x)$.由柯西收敛准则,有上述级数在$X$上一致收敛,故$f$连续,即$\mathcal{H}_{K,\mu}\subset C(X).$

又因为对任意的$x\in X$,有$K_x = \sum_{j\in J}\sqrt{\lambda_j}c_j\phi_j$,$c_j:=\sqrt{\lambda_j}\phi_j(x)$,因此有
\begin{equation*}
    \sum_{j\in J_+}c_j^2 = \sum_{j\in J}c_j^2 = \sum_{j\in J}\lambda_j\phi_j^2(x) = K(x,x)<\infty,
\end{equation*}
即$K_x\in \mathcal{H}_{K,\mu}.$此外,由式\eqref{eq: H_Kmu 内积}有
\begin{equation*}
    \langle f,K_x\rangle_{\mathcal{H}_{K,\mu}}=\sum_{j\in J_+}\sqrt{\lambda_j}a_j\phi_j(x) = f(x),
\end{equation*}
因此$\mathcal{H}_{K,\mu}$以$K$为的再生核的Hilbert空间,由RKHS的唯一性可知$\mathcal{H}_{K,\mu}=\mathcal{H_K}.$

最后证明等距同构,根据式\eqref{eq:等距同构},只需证$T_K^{1/2}$是满射即可.

对任意$f = \sum_{j\in J^+}\sqrt{\lambda_j}a_j\phi_j\in \mathcal{H}_{K,\mu}$,令$$g:=\sum_{j\in J^+}a_j\phi_j\in L^2(X,\mu)_+,$$即有$T_K^{1/2}g = f$.因此定理得证.
\end{proof}
根据定理\ref{thm:Mercer-isometry},我们有
\[
\mathcal{H}_K \;=\; H_{K,\mu} \;=\; T_K^{1/2}\big(L^{2}(X,\mu)_{+}\big).
\]
值得注意的是,$H_{K,\mu}$ 和 $T_K^{1/2}\!\left(L^{2}(X,\mu)_{+}\right)$ 的定义都依赖于
$X$ 上的 Borel 测度 $\mu$.然而,上述结果表明:实际上它们与 $\mu$ 无关,
而完全由正定核 $K$ 决定——这正是因为空间 $\mathcal{H}_K$ 完全由 $K$ 唯一定义.





\end{document}